% !TEX TS-program = pdflatex
% !TEX encoding = UTF-8 Unicode

\documentclass[11pt]{article} % use larger type; default would be 10pt

\usepackage[utf8]{inputenc} % set input encoding (not needed with XeLaTeX)
\usepackage{graphicx} % support the \includegraphics command and options
\usepackage[parfill]{parskip} % Activate to begin paragraphs with an empty line rather than an indent
\usepackage{verbatim} % adds environment for commenting out blocks of text & for better verbatim
\usepackage[hidelinks]{hyperref}
\usepackage{caption}
\usepackage{subcaption}

\title{jClustering v1.2.4 developer manual}
\author{José María Mateos \\ \texttt{jmmateos@mce.hggm.es}}

\begin{document}
\maketitle

\tableofcontents

\section{Introduction}

This document explains how to develop new clustering algorithms using the jClustering API. If you just want to use this software, please refer to the user manual (\url{https://github.com/HGGM-LIM/jclustering/blob/master/doc/user_manual.pdf?raw=true}).

Starting from version 1.2.4, the latest API documentation is attached to each release. Please download that copy of the documentation as it is the most useful resource for developers, apart from this guide.

\end{document}